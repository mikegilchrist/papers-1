\documentclass[12pt,draft]{article}


\usepackage{lineno}
\usepackage{setspace}



\begin{document}
\doublespacing
\linenumbers


\section*{Introduction}

Organisms undergo changes in phenotype due to selection on fitness relevant traits in their current environment. 
The same principle applies to the genome adapting to the intra-cellular environment.
If genomes are mixed, due to hypridization or introgression events, the part of the genome new to the current intra-cellular environment experiences added selection pressure to adapt to the new environment.

\textit{Lachancea kluyveri} is the earliest diverging lineage of the known \textit{Lachancea} clade and diverged from the \textit{Saccharomyces cerevisiae} lineage prior to their whole-genome duplication 100 Ma ago (Wolfe and Shields 1997). 
However, \textit{L. Kluyveri} and \textit{S. cerevisiae} share a similar life cycle.
Mention that kluyveri lost ability for haploid reproduction due to lack of mating type switching, is it important ???
\textit{L. Kluyveri} experienced an introgression of 1Mb about XXX ago. 
The introgression shows a difference in average GC-content of \~ 13 \% compared to the rest if the \textit{L. Kluyveri}.

Patterns of genomic variation help understanding evolution, but are mostly explored between and within species and populations.
Intra-genomic variation due to differential mutation, selection or hybridization is often ignored.
The analysis differences in patterns of codon usage allows us to explore differential evolution within a genome and differentiate between the effects of mutation and selection.

We asked if the difference in GC-content is reflected in differences in codon usage bias.
We analyzed the two genomic regions in the \textit{L. Kluyveri} genome together and separately.
Our results indicate that the introgressed region shows a different signature of codon usage bias.
We employed a mechanistic model that allowed us to distinguish between effects of mutation bias, or bias in selection for translation efficiency.
Further inspection showed that mutation bias is dominating the observed difference, while differences in selection bias between codons is mostly in agreement between the two genomic regions. 

Various attempts have been made to explain why C-Left shows a strong difference in GC-Content.
Here, we analyze C-Left and explore signals of intra-genomic heterogeneity in mutation and selection between C-Left and the rest of the genome using signals of codon usage bias.
The differential usage of synonymous codons is shaped by differences in nucleotide stability, and selection for translation efficiency and error reduction.
Nucleotide stability varies, and introduces a mutation bias between synonymous codons, prevalent in low expression genes with weak selection for translation efficiency.
In contrast, selection for translation efficiency is strong in highly expressed genes, allowing for the estimation of differences in selection bias between synonymous codons.

Selection on codon usage is assumed to be associated with the available tRNA pool in a cell. 
Codons matching more abundant tRNA species are assumed to minimize ribosome pausing due to reduced waiting time for a correct tRNA. 
Differing codon usage bias in the introgressed region can either be caused by differences in mutation bias, relating to XXXX, or differences in selection for translation efficiency, indicate a shift in the available tRNA pool.
Differences in mutation bias should persist longer due to the absence of selection.   


\begin{itemize}
	\item more general introduction, but into what? Introgression, Yeast, cub, adaptation, ...
	\item Introgression in Kluyveri, effects of introgressions, GC-content difference
	\item Codon Usage bias reflective of cellular environment, e.g. available tRNA pool
	\item Mechanistic model allows to distinguish between effects of selection and mutation on current observed state.
\end{itemize}
	
	
\section*{Results}
The striking difference in GC-content between C-Left and the rest of the \textit{L. kluyveri} genome lead us to the question whether the observed differences in GC-content are an indicator for intra-genomic variation in codon usage bias.
We hypothesised that the observed GC-content variation in the \textit{L. kluyveri} genome and the codon usage bias are independent, and we will not observe intra-genomic variation in codon usage bias.
Analysis of the \textit{L. kluyveri} genome with ROC SEMPER revealed low variation in predicted protein synthesis rate $\phi$, indicating strong disagreement in genome wide parameters and hinting at intra-genomic variation.
  
To test whether C-Left shows distinguished codon usage patterns, we separated C-Left from the rest of the genome and analyzed both parts of the genome separately.
Separation of the genome was strongly favored by model selection ($\Delta AIC: 81,504$) indicating differing patterns of codon usage bias between C-Left and the rest of the genome.

\subsection*{\textit{L.kluyveri} displays differential codon usage bias}
The separation of the genome into two independent part improved our ability to predict protein production rate $\phi$ (H0: $\rho = 0.59$ to H1: $\rho = 0.69$).
Using the whole genome, $\phi$ almost perfectly separated the genes of C-Left and the rest of the \textit{L. kluyveri} genome. 
The separation resulted in genes that are not located on C-Left to be estimated to be medium to high expressed while genes located on C-Left were estimated to be low expression genes. 
This separation of C-Left from the rest of the genome resulted into low and high expression genes resulted in mutation bias ($\Delta M$) being only informed by genes on C-Left while selection for translation efficiency ($\Delta \eta$) being only informed by genes not located on C-Left. 
In addition, the standard deviation of the $\phi$ distribution of the whole genome was greatly decreased compared to the separated dataset (H0 : $s_{\phi}$ = 0.2 to H1 : $s_{\phi}$ = 1).

\subsection*{Differences in mutation bias are greater than in selection bias}
The hypothesis best supported by the data allows us to attribute differences in GC-content mostly to mutation bias. 
A greater difference in mutational bias between the two genomic regions is revealed while selection for translation is mostly consistent between the two genomic regions. 
The comparison of the estimated mutation bias ($\Delta M$) resulted in strong disagreement in codons being favored by mutation ($\rho = -0.31$). 
Most codons that show a positive mutation bias in C-left are disfavored by mutation in the remainder of the genome.  
Furthermore, the strength of mutation bias is increased in C-left, reflected in the larger range of inferred $\Delta M$ values, revealing not only differences in mutation bias, but also showing that mutation bias in C-Left is stronger than in the rest of the genome. 
Estimates for $\Delta \eta$ showed a higher consistency between the two genomic regions ($\rho = 0.68$), indicating that - generally - the same codons are favored throughout the genome. 
The difference in time scale of protein synthesis rate $\phi$ between C-Left and Main, does not allow to draw any conclusions from the observed differences in range of the inferred $\Delta \eta$ parameter.

\subsection*{Origin of C-Left is unknown}
We determined the codon usage patterns of related yeast species to explore if one of the species or genus explored shows a similar codon usage pattern.
Agreement in codon usage patterns between C-Left and other yeast species would indicate a candidate for the origin of C-Left.
The chosen X number of yeast species displayed high agreement in codon usage, but differed in codon usage from C-Left.

Analyses of codon usage patterns of various yeast species showed that differences of codon usage is small between yeasts.
The striking difference in codon usage in \textit{L. kluyveri} not found between other yeast species lead us to explore the possibility of an origin outside the XXX clade.
INSER STUDY ABOUT BRANCH LENGTH HERE!!! 

\subsection*{Rate of evolution of estimated parameters}
We determined the rate at which the $\Delta eta$ and $\Delta M$ parameters evolved in the set of yeast species chosen.
We found that the rate of evolution is low.

$\Delta \eta$ is the strength of selection relative to the strength of drift ($sN_e$).


\begin{itemize}
	\item Question: GC-content variation causes difference in CUB, CUB difference driven by different selection or mutation environment of original host
	\item C-left has lower \% of high expression genes
	\item Observing variation in codon usage bias in kluyveri genome
	\item differences in mutation greater than in selection
	\item ignoring variation effects information extracted from the data
	\item no lachancia with similar cub known, most yeast have similar cub.
\end{itemize}

\section*{Discussion}

Attempts to elude on the unknown history of C-Left have been plenty full. 
Our comparison of codon usage patterns across various yeast species revealed no promising candidate as origin of C-Left.
However, our approach can be used to quickly screen for candidates showing similar codon usage patterns.

The evolution of codon usage is driven by the main forces of evolution: mutation, selection and drift.
The insertion of a genomic region into a new host environment is expected to lead to adaptation in the inserted region.
Neutral change is slow as no selection is amplifying the proliferation of introduced changes.
Therefore we expect that differences in mutation bias, driving codon usage in low expression genes, to decay slower than differences in selection bias.
C-Left shows an access number of genes with low evolutionary mean expression.
The slow decay of mutation bias is supported by an excess of polymorphisms of low frequency, like expected under a model of neutral evolution (Friedrich 2014).

The decay of bias in selection for translation efficiency, however, is depended on the effective population size $N_e$.  
$N_e$ is estimated to be on the order of $10^x$ for yeast species, implying that genetic drift can be assumed to be weak relative to selection.
Our findings indicate that the origin of C-Left will show similarities in mutation bias, but will likely differ in their selection bias.


\begin{itemize}
	\item Faster decay of selection bias, versus mutation bias
	\item expected number of mutations, C-Left not at equilibrium
	\item low amount of high expression genes lead to longer persistence of GC-difference?
\end{itemize}


\end{document}