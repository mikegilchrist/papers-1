\documentclass[12pt]{article}

\usepackage{xspace}
\usepackage{lineno}
\usepackage{setspace}
\usepackage{graphicx}
\usepackage{subfigure}
\usepackage{float}
\usepackage{color}
\usepackage{caption}
\usepackage[margin=1in]{geometry}
\usepackage{epstopdf}
\usepackage{natbib}
\usepackage{amsmath}

\begin{document}
\doublespacing
\linenumbers


\newcommand{\Lik}{\ensuremath{\text{\emph{L}}}\xspace}
\newcommand{\selacDMS}{\emph{SelAC}+DMS\xspace}
\newcommand{\phydms}{\emph{phydms}\xspace}
\newcommand{\selac}{\emph{SelAC}\xspace}
\newcommand{\ecoli}{\textit{E. coli}\xspace}
\newcommand{\gy}{\emph{GY94}\xspace}

\noindent RH: LANDERER ET AL.--- Estimating genetic load
% put in your own RH (running head)
% for POVs the RH is always POINT OF VIEW
\bigskip
\medskip
\begin{center}

% Insert your title:
\noindent{\Large \bf Phylogenetic model of stabilizing selection is more informative about site specific selection than extrapolation from laboratory estimates}
\bigskip


\noindent{C\textsc{EDRIC} ~{L\textsc{ANDERER}}$^{1,2,*}$,
B\textsc{RIAN} C.~ {O\textsc{MEARA}}$^{1,2}$,
\textsc{AND}
M\textsc{ICHAEL} A.~{G\textsc{ILCHRIST}}$^{1,2}$}

\end{center}

\vfill

{\small
\noindent$^{1}$Department of Ecology \& Evolutionary Biology, University of Tennessee, Knoxville, TN 37996-1610\\
\noindent$^{2}$National Institute for Mathematical and Biological Synthesis, Knoxville, TN 37996-3410\\
\noindent$^{*}$Corresponding author. E-mail:~cedric.landerer@gmail.com
}

\vfill
\centerline{Version dated: \today}
\vfill
\newpage


\section*{Introduction}
\begin{itemize}
	\item Numerous attempts to incorporate selection into phylogenetic models have been made.
	\begin{itemize}
		\item Phylogenetic inference of sequence relationship was long only focused on substitution rates and fixation probabilities.
		\item However, the importance of site specific equilibrium frequencies has long been noted.
		\item Models of site specific equilibrium frequencies tend to be unfeasible as they are very parameter rich.
		\item Independent fitness estimates have potential to greatly reduce number of parameters estimated from phylogenetic data.
		\item Incorporating selection from experimental sources therefore seems like an attractive option.
		\begin{itemize}
			\item site specific amino acid preferences acknowledge the heterogeneity of selection along the protein sequence.
			\item It allows for the fitting complex site specific models to smaller data sets.
			\item DMS allows to estimate empirical selection on amino acids for large amount of mutations on a single experiment.
		\end{itemize}
		\item Value of DMS depends on many factors like initial library of mutants and applied selection.
		\begin{itemize}
			\item Extensive mutation libraries lead to heterogeneous competing population.
			\item DMS experiments are limited to proteins and organisms that can be manipulated under laboratory conditions.
			\item Greatly limiting application of experimentally informed phylogenetic models.
		\end{itemize}
	\end{itemize}
	\item Even when empirical selection estimates are available, their application for phylogenetic inference is questionable.
	\item In this study, we compared experimentally inferred site specific selection to inform phylogenetic models to a site specific model of stabilizing selection.
	\item We assessed model fit of two codon models of site specific stabilizing selection with (\selacDMS, \phydms) and without (\selac) experimentally inferred selection and compared the models fits to 227 other codon and nucleotide models.
	\begin{itemize}
		\item We used the class A $\beta$-lactamase TEM found in gram-negative bacteria like \ecoli for which empirical selection estimates are available.  
		\item The applied selection pressure was limited to ampicillin and focused on the sequence variant TEM-1.
		\item TEM can confer resistance to a wide range of antibiotics.
		\item Models fits informed by experimentally inferred selection improve model fit over conventional codon and nucleotide models but can be improved upon using a hierarchical phylogenetic framework of stabilizing selection: \selac.
		\item Simulations highlight the inadequacy of experimentally inferred selection.
		\item Comparison between \selac and empirical estimates of selection show that they are comparable when site specific selection is captured adequately by the experiment.
		\item Furthermore, we show that extrapolating experimentally inferred selection between homologous proteins (TEM and SHV) can be inadequate.
	\end{itemize}
\end{itemize}

\section*{Results}
\subsection*{Site Specific Stabilizing Selection on Amino Acids Improves Model Fit}
\begin{itemize}
	\item We evaluated here model fits of site specific selection and 227 other codon and nucleotide models to 49 observed TEM sequences.
	\begin{itemize}
		\item All three models of site specific selection improved model fit.
		\item Number of parameters estimated from phylogenetic data differs between \selac, and \selacDMS and \phydms, resulting in slightly worse AICc for \selac.
		\item However, \selac outperforms \phydms (Table \ref{tab:AIC}).
	\end{itemize}
\end{itemize}

\begin{table}[h]
  \centering
  \caption{Model selection, shown are the three models of stabilizing site specific amino acid selection (\selac, \selacDMS, \phydms) and the best performing codon and nucleotide model \citep{GoldmanAndYang1994, zharkikh1994}. 
  Reported are the log-likelihood $\log(\Lik)$, the number of parameters estimated $n$, AIC, $\Delta$AIC, AICc, and $\Delta$AICc values.
  See Table X for results from all models we tested.}  
  \begin{tabular}{lrrrrrr}
    \hline
    Model		& $\log(\Lik)$ & n & AIC & $\Delta$AIC & AICc & $\Delta$AICc\\ \hline 
    \selacDMS 		& -1768 & 111& 3758& 14	& 3760  & 0\\
    \selac		& -1498 & 374& 3744&  0	& 3766  & 6 \\
    \phydms 		& -2061 & 102& 4326& 582& 4328 & 568\\
    \emph{SYM}+R2 		& -2230 & 102& 4663& 919& 4694 & 934 \\
    \gy+F1X4+R2 		& -2243 & 102& 4690& 946& 4821 & 1061 \\ \hline
  \end{tabular}
  \label{tab:AIC}
\end{table}

\begin{itemize}
	\item We observe differences in topology.
	\begin{itemize}
		\item \selac is to slow for a topology search, therefore unclear if the difference in topology can be attributes to the experimentally inferred selection.
		\item \gy is outperformed by several nucleotide model e.g. \emph{SYM}+R2, potentially indicating that negative frequency dependent selection is inappropriate for TEM.
		\item Results indicate shift in the evolution from the tips (\selac) to internal branches (\selacDMS, \phydms, \gy).
	\end{itemize}
\end{itemize}

\subsection*{Assessing Adequacy of Laboratory and \selac Inferences of Site Specific Selection}
\begin{itemize}
	\item Assessing model adequacy as sequence similarity sequence of selectively favored amino acids and observed consensus sequence.
	\begin{itemize}
		\item Experimentally inferred selection is inconsistent with observed sequences.
		\item Experimentally inferred sequence of selectively favored amino acids has only $52 \%$ sequence similarity with the observed consensus sequence.
		\item \selac inferred sequence of selectively favored amino acids has $99 \%$ sequence similarity with the observed consensus sequence.
		\item The average sequence similarity between the 49 observed sequences is $98 \%$.
	\end{itemize}
	\item Assessing model adequacy as genetic load.
	\begin{itemize}
		\item Simulations under experimentally and \selac inferred selection were used to establish a baseline expectation.
		\item Assuming the site specific selection estimated by DMS, the observed TEM sequences represent an average sequence specific genetic load of $17.88$ and an average site specific load of $0.065$.
		\item Simulated sequences showed an average sequence specific load of $6.68$ and an average site specific genetic load of $0.025$
		\item Assuming the site specific selection estimated by \selac, the observed TEM sequences represent an average sequence specific genetic load of $6.4\times10^{-5}$ and an average site specific load of $2.4\times10^{-7}$.
		\item Simulated sequences showed an average sequence specific load of $1.3\times10^{-5}$ and an average site specific genetic load of $4.8\times10^{-8}$.
	\end{itemize}
\end{itemize}

\subsection*{Comparing Laboratory and \selac Inferences of Site Specific Selection}
\begin{itemize}
	\item Distribution of genetic load differs between experimentally inferred site specific selection and \selac inferred site specific selection.
	\begin{itemize}
		\item Assuming the site specific selection estimated by DMS, 111 sites do not carry a genetic load.
		\item Assuming the site specific selection estimated by \selac, 207 sites do not carry a genetic load.
		\item The selection estimates from DMS and \selac agree for 107 sites that no genetic load is carried.
		\item Thus, for 100 sites \selac does not estimate a genetic load but DMS does, while the inverse is true for four sites.
		\item For the 52 sites where both, DMS and \selac, estimate a non-zero genetic load we find a correlation of $\rho = 0.247$, explaining $6 \%$ of the variation in the empirical selection estimates.
	\end{itemize}
\end{itemize}

\subsection*{Comparing \selac Inferences of Site Specific Selection for Homologous Sequences TEM and SHV}
\begin{itemize}
	\item Site specific genetic load for TEM and SHV is not correlated ($\rho = 0.006$) and , despite similar $\alpha_G$
	\begin{itemize}
	 \item Exclusing site with no genetic load and calculating the correlation on the log scale lead to a correlation coefficient of $\rho = 0.22$. 
	\end{itemize}
	\item Greatest difference is observed in the physicochemical properties, specifically $\alpha$.
	\item No significant differences are observed in average genetic load between secondary structure elements (Table \ref{tab:selection}).
\end{itemize}

\begin{table}[h]
  \centering
  \caption{Efficacy of selection ($G$) and genetic load for TEM and SHV, and separated by secondary structure. $G$ was estimated as a truncated variable with an upper bound of 300.}
  \begin{tabular}{llrrrrr}
    \hline
    & & & \multicolumn{2}{c}{$G$} & \multicolumn{2}{c}{Genetic Load $L_i$} \\ 
    Protein & Secondary Structure & \# Residues	& \multicolumn{1}{c}{Mean} & \multicolumn{1}{c}{SE} & \multicolumn{1}{c}{Mean} & \multicolumn{1}{c}{SE} \\ \hline 
    TEM	&		& 263 & 219.3 & 7.5  & $15.9\times10^{-8}$ & $6.5\times10^{-8}$ \\
    &Helix 		& 113 & 206.1 & 12.4 & $17.5\times10^{-8}$ & $13.1\times10^{-8}$ \\
    &$\beta$-Sheet 	&  48 & 238.6 & 15.8 & $ 6.8\times10^{-8}$ & $2.9\times10^{-8}$ \\
    &Unstructured 	& 102 & 224.8 & 11.4 & $18.6\times10^{-8}$ & $8.1\times10^{-8}$ \\
    &Active/Binding Sites 	&   5 & 202.6 & 62.2 & $0.01\times10^{-8}$& $0.01\times10^{-8}$ \\ \hline
    
    SHV&		& 263 & 244.9 & 6.8  & $4.0\times10^{-8}$ & $1.9\times10^{-8}$ \\
    &Helix		& 102 & 234.6 & 11.5 & $7.3\times10^{-8}$ & $4.8\times10^{-8}$ \\
    &$\beta$-Sheet 	&  66 & 253.1 & 12.8 & $2.1\times10^{-8}$ & $1.1\times10^{-8}$ \\
    &Unstructured	&  95 & 224.7 & 11.4 & $1.5\times10^{-8}$ & $0.6\times10^{-8}$  \\
    &Active/Binding Sites	&   5 & 239.9 & 60.0 & $1.5\times10^{-8}$ & $1.5\times10^{-8}$ \\ \hline
  \end{tabular}
  \label{tab:selection}
\end{table}


\section*{Discussion}
\begin{itemize}
	\item We evaluated how well experimental selection estimates from DMS experiments explain natural sequence evolution and compared it to a novel phylogenetic framework, SelAC.
	\begin{itemize}
		\item Previous work has shown that DMS selection estimates can improve model fit over classical approaches like GY94 and our work confirms this.
		\item Model selection favored the SelAC model fit and the corresponding fitness estimates over the DMS estimates using both, SelAC and phyDMS (Table \ref{tab:AIC}).
	\end{itemize}

	\item Adequacy of the DMS selection has previously not been assessed.
	\begin{itemize}	
		\item The amino acid with the cumulative highest fitness experimentally estimated with DMS only has $49 \%$ concordance with the observed alignment.
		\item In contrast, the SelAC estimate has $99 \%$ concordance (Figure \ref{fig:sim_seqs_cons}). 
		\item Estimates of selection coefficients do not represent evolution.
 		\begin{itemize}
			\item Due to artificial selection environment; Heterogeneous population, very large $s$. 
			\item Only one antibiotic used, maybe a mixture of antibiotics would better reflect natural evolution.
			\item Lack of repeatability between labs introduces further problems (Firnberg et al 2014 vs. Stifler et al. 2016).
		\end{itemize}
	\end{itemize}

	\item Assuming that the DMS selection inference adequately reflects natural evolution, the observed TEM sequences are either mal-adapted or where unable to reach a fitness peak.
	\begin{itemize}
		\item \textit{E. coli} has a large effective population size, estimates are on the order of $10^8$ to $10^9$ (Ochman and Wilson 1987, Hartl et al 1994).
		\item The large $N_e$ would allow \textit{E. coli} to effectively "explore" the sequence space, thus suggesting that the TEM sequences are mal-adapted according to the DMS estimates.
		\item Our simulations of sequence evolution with various $N_e$ values and the DMS fitness values in contrast show that we would expect higher adaptation even with much smaller $N_e$ (Figure \ref{fig:dms_sim}).
	\end{itemize}

	\item Estimates of selection coefficients do not represent evolution.
 	\begin{itemize}
		\item Due to artificial selection environment; Heterogeneous population, very large $s$. 
		\item Only one antibiotic used, maybe a mixture of antibiotics would better reflect natural evolution.
		\item Lack of repeatability between labs introduces further problems (Firnberg et al 2014 vs. Stifler et al. 2016).
		\item Still better than models without site specific equilibrium frequencies.
	\end{itemize}

	\item DMS estimates of the observed TEM variants predict them to be mal-adapted while SelAC predicts most TEM variants to be well adapted.
	\begin{itemize}
		\item Given \textit{E. coli}'s large effective population size, the efficacy of selection should be very large.
		\item We therefore expect the observed sequence variants to be at the selection-mutation-drift barrier, which in turn can expected to be near the optimum.
		\item We find the majority of sequences near the optimum, therefore the SelAC estimates are consistent with theoretical population genetics results.
		\item In contrast, finding strong selection against the observed TEM variants indicates that DMS is not consistent with theoretical population genetics expectations.
		\item This is consistent when thinking about that DMS only reflects the selection on the TEM sequence with regards to one antibiotic, which seems appropriate to model selection in modern hospital environments but not when the interest lies in the natural evolution of TEM.
	\end{itemize}

	\item We find that SelAC produces similar selection against the observed TEM variants  if we assume the fitness peaks (optimal AA) that are estimated by DMS.
	\begin{itemize}
		\item This shows that DMS and SelAC can provide consistent estimates of selection against amino acids.
		\item SelAC has the advantage that it can be applied to any protein coding sequence alignment.
		\item This removes the need for extrapolation e.g. from TEM to SHV.
	\end{itemize}

	\item SelAC has the advantage that it can be applied to any protein coding sequence alignment.
	\begin{itemize}
		\item This removes the need for extrapolation e.g. from TEM to SHV.
	\end{itemize}

	\item Difference in selection parameters between TEM and SHV indicate that extrapolation is not a good idea.
	\begin{itemize}
		\item The difference in the site specific strength of selection shows that TEM and SHV are facing different selection pressures.
		\item this is also highlighted by the differences in physicochemical weightings between the two proteins.
	\end{itemize}

	\item SelAC outperforms DMS, but is not without flaws itself
	\begin{itemize}
		\item Like DMS and most phylogenetic models, SelAC assumes site independence.
		\item SelAC is a model of stabilizing selection, in contrast to e.g. GY94 which is a model of frequency dependent selection.
		\begin{itemize}
			\item Since TEM plays a role in the chemical warfare with conspecifics and other microbes, some sites may be under negative frequency dependent selection.
		\end{itemize}
		\item SelAC assumes the same G distribution across all sites.
		\begin{itemize}
			\item Different G distribution for each type of secondary structure
			\item active sites may not follow distribution.
		\end{itemize}
		\item SelAC assumes that selection is proportional to distance in physicochemical space. 
		\begin{itemize}
			\item We used Grantham (1974) properties, however many other distances are available which may an even better model fit.
		\end{itemize}
	\end{itemize}
	\item Low sequence variation in the TEM may be cause for concern as it could be misinterpreted by the model as stabilizing selection because of the short branches.
	\begin{itemize}
		\item However, provided our simulations support that TEM is actually under stabilizing selection
	\end{itemize}

	\item In conclusion, DMS experiments have been proposed to supplement information on selection on amino acids in phylogenetic studies.
	\begin{itemize}
		\item This study shows that information on selection can be extracted from alignments of protein coding sequences.
		\item This highlights the limitations of DMS to explain natural evolution.
	\end{itemize}
\end{itemize}



\end{document}







