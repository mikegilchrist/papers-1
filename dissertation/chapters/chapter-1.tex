\chapter{Introduction} 
\label{ch:introduction}

Mathematical and statistical have long been used to describe or summarize observations in genetics and genomics.
Mostly without addressing the underlying biological mechanisms mutation, selection, and drift shaping DNA sequences, but as phenomelogical description.
However, as researchers learn more about the underlying processes and more genetic and genomic data is available, the mathematical descriptions that allow us to extract information from this data have to keep up.
For example, after the unreaveling the degenerate genetic code by \citet{MatthaeiAndNirenberg1961,NirenbergAndMatthaei1961,Maxwell1962,LederAndNirenberg1964}, and many others, researchers noticed that synonymous codons are not found in uniform proportions \citep{fitch1976,grantham1980,ikemura1981,grantham1981,sharp1988}.
Models of codon usage, however, where long purley describtive and heuristic \citep{ikemura1981,BennetzenAndHall1982,sharp1987,wright1990}.
Similarly, phylogenetic models have long been phenomelogical \citep{JukesAndCantor1969,Dayhoff1978,Kimura1980,felsenstein1981,Altschul1991}, describing the rate at which one state is transformed in another, without regards for the fundamental forces of evolution mutation, selection, and drift.
\citet{ZuckerkandlAndPauling1962} described the distance between hemoglobin proteins and proposed that the evolution of proteins is constant over time and between lineages before the genetic code was fully deciphered.
This dissertation is therefore focues on the application of mechanistic models rooted in first principles to protein coding sequences.

A wide variety of information is stored in protein and protein coding sequences, e.g. structure \citep{anfinsen1973}, mutation bias \citep{ShahAndGilchrist2011, gilchrist2015}, protein synthesis rate \citep{gilchrist2007}. 
Mechanistic models can be used extract these informations and to study the relative strength of mutation, selection, and genetic drift leading to observed sequences.

Understanding the contributions of mutation, selection and drift to the evolution of observed sequences allows us to use this information to learn more about these seqeunces.


%INTRODUCTION
% - Mutation, selection, drift are three main forces of evolution
% - Selection can only act on standing variation \citep{LuriaAndDelbruck1943}
% - Protein peptide sequence contains information e.g. Structure \citep{anfinsen1973}
% - Fitness effects of mutations, synonymous and non-synonymous
% -- Protein production is expensive \citep{warner1999,AkashiAndGojobori2002}
% - phylogenetics, rate of evolution \citep{ZuckerkandlAndPauling1962}

%MECHANISTIC MODELS
% - Mechanistic model can extract information on mutation and selection that describtive ones can not
% -- CAI, indifferent to mutation \citep{sharp1987}
% -- F_opt \citep{ikemura1981}
% -- CBI \citep{BennetzenAndHall1982}
% - Phylogenetic models

%INTROGRESSION
% -

%PROTEIN FUNCTIONALLITY